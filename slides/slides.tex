\documentclass[t,11pt,hyperref={
  %pdfpagemode=FullScreen,
  pdftitle = {gearshifft},
  pdfsubject = {gearshifft},
  %linktocpage=true,
  pdfborder={0 0 0},
  colorlinks=true,
  urlcolor=red,
  citecolor=red,
  linkcolor=red,
  pdfauthor={Peter Steinbach, Matthias Werner}
  }
]{beamer}
\usetheme{custom}
\def\resetbeamertemplate{\setbeamertemplate{background canvas}{ }}
\let\Tiny=\tiny
%\usepackage{lmodern}
\usepackage{tikz}
\usetikzlibrary{matrix}
\usetikzlibrary{calc}
\usetikzlibrary{positioning}
\usetikzlibrary{arrows}
\usetikzlibrary{trees}
\usetikzlibrary{backgrounds}
\usetikzlibrary{mindmap}
\usetikzlibrary{shadows}

\usepackage[utf8]{inputenc}
\usepackage[T1]{fontenc}
\usepackage{csquotes}
\usepackage{ifthen}
\usepackage[capitalize]{cleveref} % after hyperref

\usepackage{amsmath,amstext,amsthm,array,booktabs}
\usepackage{caption}
\usepackage{xcolor}
\usepackage{graphicx}
\usepackage{subfig}
\graphicspath{{../}}
\usepackage{colortbl}
\usepackage{listings}

\usepackage[binary-units=true,
            locale=US,
            ]{siunitx}

%%%%%%%%%%
\providecommand\thispdfpagelabel[1]{}


\lstset{literate={~} {$\sim\,$}{1}}
\lstset{upquote=true}
\lstset{language=C++,
  basicstyle=\small\fontfamily{fvm}\selectfont,%\ttfamily,
% 	columns=fullflexible,
  columns=fixed,
  keepspaces=true,
%       keywordstyle=\color{black}\bf,
%       stringstyle=\color{red}\ttfamily,
  showstringspaces=false,
  commentstyle=\color[rgb]{0.3,0.3,0.3}\textit,
%       morecomment=[l][\color{gray}]{\#},
  identifierstyle=\color{black},
  morekeywords={size_t},
  emph={__global__,__device__},
        emphstyle=\textit,
        numbers=left,
  numberstyle=\small\color{gray},
  breaklines=true,
  numbersep=5pt,
  xleftmargin=.19in,
}
% %%%%%%%%%%%%%%%%%%%%%%%%%%%%%%%%%%%%%%%%%%%%%%%%%%%%%%%%%%%%%%%%%%%%%%%%%%%%%%%

% \mode<presentation>{%
% \setbeameroption{show notes}
% }

\newcolumntype{M}{>{$\displaystyle}l<{$\vspace{1ex}}}
\title[gearshifft]{\texorpdfstring{%
    gearshifft\\[.5em]\Large The FFT Benchmark Suite\\ for Heterogeneous Platforms}}

\author[Steinbach, Werner]{\texorpdfstring{%
    \begin{minipage}[t]{.49\textwidth}%
      \centering%
      Peter Steinbach\\[.5em]
      {\footnotesize{Max Planck Institute of Molecular Cell Biology and Genetics\\
          Dresden, Germany}}\\
      {\small{\url{steinbac@mpi-cbg.de}}}
    \end{minipage}%
    \begin{minipage}[t]{.49\textwidth}%
      \centering%
      Matthias Werner\\[.5em]
      {\footnotesize{Center for Information Services and High Performance Computing\\
          TU Dresden, Germany}}\\
      {\footnotesize{\url{Matthias.Werner1@tu-dresden.de}}}
    \end{minipage}}{The Author}%
}

\date{June 20, 2017}

%%%%%%%%%%%%%%%%%%%%%%%%
\tikzset{invisible/.style={opacity=0},
      visible on/.style={alt=#1{}{invisible}},
      alt/.code args={<#1>#2#3}{%
        \alt<#1>{\pgfkeysalso{#2}}{\pgfkeysalso{#3}} % \pgfkeysalso doesn't change the path
      }
    }
\tikzset{class/.style={inner sep=5pt,font=\footnotesize}}
\newcommand{\pclass}[5][]{
\ifthenelse { \equal {#1} {} }
 {\node[class] (#5) at (#3,#4) {#2};}
 {\node[class] (#5) at (#3,#4) {%
\begin{tabular}{c}\scriptsize{<<#1>>}\\{\textbf{#2}}\end{tabular}%
};}
}
\newcommand{\pclassfill}[6][]{
\ifthenelse { \equal {#1} {} }
 {\node[class,rounded corners,fill=#6] (#5) at (#3,#4) {#2};}
 {\node[class,rounded corners,fill=#6] (#5) at (#3,#4) {%
\begin{tabular}{c}\scriptsize{<<#1>>}\\{\textbf{#2}}\end{tabular}%
};}
}

% https://tex.stackexchange.com/questions/263345/listings-and-verbatim-like-environments-in-a-tikz-node-using-newcommand
\newsavebox\mybox

\newcommand{\tkzgearshifft}{
%align=center,rounded corners,inner sep=5pt,rectangle,draw,
\begin{tikzpicture}
\tikzset{gr1/.style={}}
\tikzset{bts/.style={draw,circle,inner sep=3pt,fill=white}}
\tikzset{btc/.style={draw,circle,inner sep=3pt,fill=black}}

% fft parameters
\visible<4>{
\begin{scope}[yshift=3.75cm,xshift=-5.5cm,every node/.style={anchor=west,align=left,font=\small,draw,rectangle,rounded corners,fill=white}]

\node[fill=green!50] at (0,0) {cuFFT};
\node at (0.2,-0.4) {float, \ldots};
\node at (0.4,-0.8) {1024x1024, \ldots};
\node[fill=black,text=white] (mplc) at (0.6,-1.2) {Inplace\_Real, \ldots};

\end{scope}
}

% tree
\visible<5->{
\begin{scope}[yshift=3.5cm,xshift=-3.6cm]
\node[bts,fill=green!50] (b0) at (0,0) {};
\node[bts] (b10) at (-0.5,-0.6) {}; \draw (b10) -- (b0);
\node[bts] (b11) at (0.5,-0.6) {}; \draw (b11) -- (b0);
\node[btc] (b20) at (-0.75,-1.3) {}; \draw (b20) -- (b10);
\node[btc] (b21) at (-0.25,-1.3) {}; \draw (b21) -- (b10);
\node[btc] (b22) at ( 0.25,-1.3) {}; \draw (b22) -- (b11);
\node[btc] (b23) at ( 0.75,-1.3) {}; \draw (b23) -- (b11);
\node[font=\small] at (0, 0.4) {{\textbf{Boost}} Test Suites};
\node[font=\small] at (0,-1.7) {{\textbf{Boost}} Test Cases};

\end{scope}}

% 
\begin{scope}[xshift=1.75cm]
\begin{scope}

\visible<3->{
\pclass{{\textbf{Benchmark}}}{-2}{4}{b}
\pclass[Functor]{BenchmarkSuite}{-2}{3.2}{bs}
\pclass[Functor]{BenchmarkExecutor}{-2}{2.1}{be}
}
\pclassfill[Functor]{FFT}{-2}{1.0}{fft}{yellow}

\end{scope}
\begin{scope}

\pclassfill[Realisation]{Context}{1.5}{2.5}{ctx}{green!50}
\pclassfill[Realisation]{FFTClient}{1.5}{1.1}{impl}{green!50}
\visible<2->{
\pclass[Singleton]{Application}{1.5}{4}{app}
% \draw[rounded corners, dashed] (ctx.north west) rectangle (impl.south east);
% \draw[rounded corners] (ctx.north west) rectangle (ctx.south east);
% \draw[rounded corners] (impl.north west) rectangle (impl.south east);
\draw[rounded corners] (app.north west) rectangle (impl.south east);
}
\end{scope}
\end{scope}

\visible<-5>{
\matrix[
 minimum height=1.5em,
 matrix of nodes,
 row sep=-\pgflinewidth,
 ampersand replacement=\&,
 column sep=-\pgflinewidth,
 text depth=2.5ex,
 text height=1.5ex,
 text width=3.0em,
 align=center,
 nodes in empty cells,
 row 1/.style={nodes={fill=black!15,draw=black!40,text=black,thick,rectangle,draw,minimum width=2.5em,font=\scriptsize}}
]
(mf) at (0,-0.5) {
allocate \&
init\linebreak forward \&
|[gr1]| upload \&
|[gr1]| execute\linebreak forward \&
init\linebreak inverse \&
|[gr1]| execute\linebreak inverse \&
|[gr1]| download \&
destroy\\
};
}

\begin{scope}[thick]
\visible<-5>{
\draw (mf-1-1.north west) ++(-0.75em,0.5em) coordinate (ctl) -- ([xshift=0.75em,yshift=0.5em]mf-1-8.north east) coordinate (cr);
\draw[dotted] (ctl) -- ++(-2ex,0); \draw[dotted] (cr) -- ++(2ex,0);
\draw (mf-1-1.south west) ++(-0.75em,-0.5em) coordinate (cl) -- ([xshift=0.75em,yshift=-0.5em]mf-1-8.south east) coordinate (cr);
\draw[dotted] (cl) -- ++(-2ex,0); \draw[dotted] (cr) -- ++(2ex,0);
% total time
\draw[dashed,|-|] (mf-1-1.south west) ++(0,-1.5em) -- ([yshift=-1.5em]mf-1-8.south east) node[pos=0.5,fill=white,font=\small\itshape,align=center,anchor=north] {total time\\(measured separately)};
% (mf-1-1.south west) -- ++(0,-1.5em) -| (mf-1-8.south east) node[pos=0.25,fill=white,font=\small] {total time};
}

\visible<3->{
% \draw[-latex] (b) -- (bs);
% \draw[-latex] (fft) -- (ctl-|fft);
\draw[black!80] (b.south west) -- (b.south east);
\draw[black!80] (bs.south west) -- (bs.south east);
\draw[black!80] (be.south west) -- (be.south east);
\draw[-angle 60] (b) -- (app.west|-b);
}
\visible<4>{
  \draw[black!80, dashed] (b.south west) -- ++(-1em, 1.25em) -- ++(-5em,0); % test suite marker
  \draw[black!80, dashed] (bs.south west) -- ++(-1em,-1.5em) -- ++(-4em,0); % test suite marker
}

\visible<2->{
  \draw[-angle 60] (app) -- (ctx);
}
\draw[-angle 60] (impl) -- (ctx.south);
% \draw[densely dashed,-angle 90] (be.east) -| (impl.north);
% \draw[densely dashed,-open triangle 60] (impl) -- (fft.east|-impl) node[midway] (q) {};
\draw[-angle 90] (fft) -- (impl.west|-fft) node[midway] (q) {};
\visible<-5>{\draw[] (q.center) -- (ctl-|q);}
\end{scope}

\visible<6->{
  \node[align=left, anchor=north] (mpl) at (-1,-0.65) {Compile-time loops for precision and transform type, i.\,e.:\\[.5em]\usebox\mybox};
  \draw[-latex] (mpl) -- (mplc);
}
\end{tikzpicture}
}


%%%%%%%%%%%%%%%%%%%%%%%%%%%% 
\definecolor{arccl}{rgb}{0.45,0.45,0.45}
\definecolor{mc1}{rgb}{0.0, 0.0, 1.0}
\definecolor{mc2}{rgb}{0.0, 0.5, 0.5}
\definecolor{mc3}{rgb}{0.5, 0.0, 0.0}
\newcommand{\mgraymidrule}{\arrayrulecolor{arccl}\midrule\arrayrulecolor{black}}
%%%%%%%%%%%%%%%%%%%%%%%%%%%%%%%%%%%%%%%%%%%%%%%%%%%%%%%%%%%%%%%%%%%%%%%%



\newcommand{\gearshifft}{\texttt{gearshifft}}
\newcommand{\fftw}{\texttt{fftw}}
\newcommand{\cufft}{\texttt{cuFFT}}
\newcommand{\clfft}{\texttt{clFFT}}
\newcommand{\nvidia}{Nvidia}
\newcommand{\mc}[1]{\lstinline!#1!}
\newcommand{\iu}{{\mathrm{i}\mkern1mu}}

%% Todo: replace dirtree with gearshifft infos
%%        what is templated, what is runtime

\begin{document}

\frame[plain]{\titlepage}

\begin{frame}{Fields of the Fast Fourier Transformation}
  \centering
  \begin{tikzpicture}
    \begin{scope}[every node/.style={align=center, anchor=north, font=\bfseries}, xscale=1.15, yscale=2.6]
      \node[fill=gray,text=white,inner sep=5pt, minimum width=3cm,minimum height=2cm,rectangle,rounded corners, anchor=center, drop shadow] at (0,-0.5) {\Huge\textbf{FFTs}};
      % https://commons.wikimedia.org/wiki/File:JPEG_compression_Example.jpg
      \node [visible on=<3->] (comprimage) at (-3.5, 0.0) {\includegraphics[width=0.25\textwidth]{jpeg-compression.jpg}\\ compression};
      % https://imagej.net/Multiview-Reconstruction
      \node[visible on=<3->] at ( 3.5, 0.0) {\includegraphics[width=0.25\textwidth]{imagej.jpg}\\ biology};
      % https://commons.wikimedia.org/wiki/File:Deep_learning.png
      \node[visible on=<2->] at (-3, 1.0) {\includegraphics[width=0.21\textwidth]{deep-learning.png}\\ machine learning};
      % https://en.wikipedia.org/wiki/Trader_%28finance%29#/media/File:Philippine-stock-market-board.jpg
      \node[visible on=<2->] at ( 3, 1.0) {\includegraphics[width=0.25\textwidth]{Philippine-stock-market-board.jpg}\\ financial math};
      % https://de.wikipedia.org/wiki/Sinc-Funktion#/media/File:Si_sinc.svg
      \node at ( 0, 1.15) {\includegraphics[width=0.25\textwidth]{sin-cos.png}\\ signal processing};
      % https://en.wikipedia.org/wiki/Interferometry#/media/File:USA.NM.VeryLargeArray.02.jpg
      \node[visible on=<4->] at (-3,-1.0) {\includegraphics[width=0.25\textwidth]{usa-nm-vla.jpg}\\ astronomy};
      % https://en.wikipedia.org/wiki/Geology_applications_of_Fourier_transform_infrared_spectroscopy#/media/File:FTIR_spectrum.jpg
      \node[visible on=<4->] at ( 3,-1.1) {\includegraphics[width=0.25\textwidth]{FTIR_spectrum.jpg}\\ geology};
      \node[visible on=<5->] at ( 0,-1.65) {\Large\textbf{\ldots}};
    \end{scope}
  \end{tikzpicture}
\end{frame}


\begin{frame}{Introduction}{Fourier Transformation}

\vfill
  \begin{equation}
    \label{eq:dft}
    \text{\textbf{DFT:} }\quad X[k] = \sum_{j=0}^{n-1} x[j]\cdot\exp\left(\frac{-2\pi \iu jk}{n}\right),\quad x,X\in\mathbb{C}^n
  \end{equation}
\vfill  
  \begin{itemize}
  \item fast implementation of discrete Fourier transform (DFT) \eqref{eq:dft}
  \item forward transform: time domain $\Rightarrow$ frequency domain
  \item factorization of $n$ and recursive decomposition yield smaller DFTs $\Rightarrow$ FFT$\sim\mathcal{O}(n\log n)$

    \begin{itemize}
    \item Cooley-Tukey: Radix-2 DFTs
    \item Stockham's formulations avoid incoherent memory access
    \item Bluestein's algorithm allows arbitrary and mixed radices
    \end{itemize}

  \end{itemize}
\vfill

\end{frame}

\begin{frame}{Introduction}{FFTs and their parameters}
  \centering  
  \begin{tikzpicture}[root concept/.style={rectangle, rounded corners},
    small mindmap]
%level 1 concept/.append style={sibling angle=20}, small mindmap]

    \path[concept color=white,text=black]
    node[concept,font=\huge\bfseries] {FFTs}
    [grow=down]
    child[concept color=red, text=white] { node[concept] {\textbf{library}\\ cuFFT, clFFT, fftw, \ldots} }
    child[concept color=green] { node[concept] {\textbf{transform}\\ real, complex} }
    %child[concept color=brown] { node[concept] {\textbf{specifics}\\ rigors} }
    child[concept color=black,text=white] { node[concept] {\textbf{radices}\\ $2^n$, mixed} }
    child[concept color=cyan] { node[concept] {\textbf{dims}\\ $1$, $2$, $3$} }
    child[concept color=gray,text=white] { node[concept] {\textbf{hardware}\\ CPU, \\GPU} }
    child[concept color=orange] { node[concept] {\textbf{memory}\\ inplace, outofplace} }
    child[concept color=blue,text=white] { node[concept] {\textbf{precision}\\ single, double, \ldots} };

  \end{tikzpicture}

  \vfill
  \pause

  \setbeamertemplate{itemize items}[triangle]
  \begin{itemize}
  \item Which FFT implementation works best on what hardware?
  \item What hardware is best for my transform configuration?
  \item What has changed between library versions?
  \item \ldots
  \end{itemize}
  \setbeamertemplate{itemize items}[circle]

\end{frame}

\begin{frame}{gearshifft}

\vfill
      \begin{center}\Large
        \href{https://github.com/mpicbg-scicomp/gearshifft}{\textbf{github.com/mpicbg-scicomp/gearshifft}}
      \end{center}

\vfill
  \begin{itemize}
  \item free and open-source benchmark suite for FFT libraries
    \begin{itemize}
    \item various conditions of FFTs can be benchmarked
    \item supports cuFFT, clFFT and fftw (mkl and hcfft planned)
    \item performs round-trip $fft^{-1}(fft(signal))$ and validation
    \end{itemize}
    \pause
    \vfill
  \item licensed under Apache License 2.0
  \item community-ready for contributions
  \item vendor independent 
  \item standardized output format
  \item open and extensible architecture written in C++14
  \end{itemize}
\vfill
\pause
\begin{center}
  \includegraphics[width=.35\textwidth]{gearshifft_logo_img_100.png}
\end{center}

\vfill
\end{frame}

\begin{frame}[fragile]{gearshifft}{Rational}

Simplified Workflow for a \textbf{single} FFT benchmark of an FFT lib\\
  (e.g. 1024-FFT real-inplace single-precision):\\[.5em]

\begin{minipage}[t]{0.34\textwidth}
\textbf{FFT/lib operations}
\begin{lstlisting}[numbers=none]
init_forward();
exec_forward();
init_inverse();
exec_inverse();
destroy_plan();
\end{lstlisting}
\end{minipage}
\begin{minipage}[t]{0.3\textwidth}
\begin{onlyenv}<3->
  \textbf{+ Timer}
  \begin{lstlisting}[numbers=none]
// ...
timer_start();
init_forward();
val = timer_stop();
// ...
  \end{lstlisting}
\end{onlyenv}
\end{minipage}
\begin{minipage}[t]{0.3\textwidth}
\begin{onlyenv}<4->
  \textbf{+ Results}
  \begin{lstlisting}[numbers=none,emph={PlanInit}]
// ...
timer_start();
init_forward();
val = timer_stop();
store(val, PlanInit);
// ...
  \end{lstlisting}
\end{onlyenv}
\end{minipage}

\vfill
\only<2>{
\begin{itemize}
\item context and data management excluded here
\item context in \gearshifft{} has application lifetime
\item round-trip FFT API is templated, specialized by library clients
\end{itemize}}

\only<3>{
\begin{itemize}
\item time FFT operations
\item \gearshifft{} templates and reuses timer objects
  \begin{itemize}
  \item developer can provide own timer class
  \item reusing timer objects minimizes latency
  \end{itemize}
\end{itemize}}

\only<4>{
\begin{itemize}
\item time values are collected
  \begin{itemize}
  \item \gearshifft{} uses \mc{std::array[<run_id>][<fft_op_id>]}
  \end{itemize}
\item \gearshifft{} stores results in csv eventually
  \begin{itemize}
  \item also contains meta-data, allocation and transfer sizes and times for context setup 
  \item statistics of benchmarks can be displayed
  \end{itemize}
\end{itemize}}

\end{frame}
  

% code used in tkzgearshifft
\begin{lrbox}{\mybox}
\begin{lstlisting}[numbers=none]
for( T_Precision : {float, double}) // compile-time
 for( extent : {32, 64x32, 16x16x16})
  for( T_FFT : {Inplace_Real, Outplace_Complex})  // compile-time
   // instantiate and run benchmark
\end{lstlisting}
\end{lrbox}


\begin{frame}[fragile]{gearshifft}{Basic Framework in C++/Boost}
\hspace*{-2em}
  \tkzgearshifft
\vspace{-3em}
\only<1>{  
  \begin{itemize}
  \item \mc{FFTClient} implements FFT workflow template class
  \end{itemize}
  }
\only<2>{  
  \begin{itemize}
  \item Application controls context object and results
  \end{itemize}
}
\only<3>{  
  \begin{itemize}
  \item Boost UTF driven generation of benchmark instances
  \end{itemize}
  }
\only<4>{\vspace{-1em}
  \begin{itemize}
  \item Boost \tikz[baseline=-4pt]{\node[draw,rounded corners,rectangle] {test suites};} are generated by lib, precision and extents, the Boost \tikz[baseline=-4pt]{\node[draw,rounded corners,rectangle,fill=black,text=white] {test case};} is a specialized FFT client
  \end{itemize}
  }
\only<5>{  
  \begin{itemize}
  \item the benchmark tree is traversed by Boost \ldots
  \end{itemize}
  }
\end{frame}

\begin{frame}[fragile]{gearshifft}{Command-line Examples}
  \begin{itemize}[<+->]
\item 
\textbf{benchmark clFFT on CPU}: load extents.csv and store results in result.csv
\begin{lstlisting}[numbers=none,language=bash]
$ ./gearshifft_clfft -f ../config/extents.csv
                     -o result.csv -d cpu
\end{lstlisting}

\item Command-line supports wildcards from Boost options
\item 
\textbf{benchmark clFFT on GPU} with $extent=1048576$, float and out-of-place real transforms
\begin{lstlisting}[numbers=none,language=bash]
$ ./gearshifft_clfft -e 1048576
                     -r */float/*/Outplace_Real
# -r <run_filter> with 
#    <Lib>/<Precision>/<Extents>/<Transform>
\end{lstlisting}
\item \href{https://github.com/mpicbg-scicomp/gearshifft/tree/master/config}{predefined extent lists} available in the gearshifft repository
\end{itemize}  
\end{frame}


\begin{frame}{gearshifft}{Libraries}
  fftw
  \begin{itemize}
  \item for CPUs only, code 20 years old
  \item not designed for modern multi-core CPUs    
  \item search-depth for optimal plan is controlled by rigor flags
  \item precomputed plans can be stored to ``wisdoms''
  \end{itemize}
  
  cuFFT 
  \begin{itemize}
  \item Nvidia's CUDA implementation of FFTs for Nvidia GPUs
  \item supports arbitrary and mixed radices
  \end{itemize}

  clFFT
  \begin{itemize}
  \item AMD's OpenCL implementation of FFTs
  \item optimized for AMD GPUs, but also works on CPUs and other compute devices
  \item supports mixed radices 2, 3, 5, 7, 11 and 13
  \end{itemize}
  
\end{frame}

\begin{frame}{gearshifft}{Setup}
\begin{table}[tbp]
  \centering
  {\scriptsize{
  \caption{Benchmark Hardware}
  \label{tab:hardware}
  \begin{tabular}{lll}
    \toprule
                        & \multicolumn{2}{c}{\textbf{Taurus}}                                         \\
                        & \multicolumn{2}{c}{HPC Cluster}                                             \\
    \midrule
    \textbf{CPU family} & Haswell Xeon               & Haswell Xeon           \\
    \textbf{CPU model } & $2{\times}$ E5-2680 v3     & $2{\times}$ E5-2680 v3 \\
    \textbf{RAM       } & \SI{64}{\gibi\byte}        & \SI{64}{\gibi\byte}    \\
    \textbf{GPU} {\scriptsize{(PCIe3.0)}} & 4x K80   & 1x P100                \\
    \textbf{GPU memory} & 4x \SI{12}{\gibi\byte}     & \SI{16}{\gibi\byte}    \\
    \textbf{GPU driver} & $367.48$                   & $375.66$               \\
    \textbf{OS}         & RHEL $6.8$                 & CentOS $7.3$       \\
    \bottomrule
  \end{tabular}}}
\end{table}

  
  \begin{itemize}
  \item 12 runs = 2 warmups + 10 warm runs
  \item CPU and GPU clock fixed to non-boost where possible
  \end{itemize}
\end{frame}


\begin{frame}{Results}{Validating gearshifft cufft vs. standalone cufft}
\begin{figure}[!htb]
  \centering
  \includegraphics[width=\textwidth]{figures/results_validate_cufft_legend.pdf}\\[-.5em]
  \subfloat[1024-point FFT]{\parbox[b]{0.46\textwidth}{%
    \includegraphics[width=0.5\textwidth]{figures/results_validate_cufft_a.pdf}\\
    \includegraphics[width=0.5\textwidth]{figures/results_validate_cufft_c.pdf}
    }\label{fig:tts_verify_a}}
  \hfill
  \subfloat[16777216-point FFT]{\parbox[b]{0.5\textwidth}{%
      \includegraphics[width=0.5\textwidth]{figures/results_validate_cufft_b.pdf}\\
      \includegraphics[width=0.5\textwidth]{figures/results_validate_cufft_d.pdf}
    }\label{fig:tts_verify_b}}
  \caption{\textit{gearshifft} (\cufft{}) vs. \textit{standalone} \cufft{} --  multiple timer objects \& one timer object (\textit{standalone-tts}) -- single-precision in-place real-to-complex round-trip FFTs on the K80.}
  \label{fig:verify_cufft}
\end{figure}
\end{frame}


\begin{frame}{Results}{Exploring fftw plan-rigor flags}
\begin{figure}[!htbp]\vspace{-1em}
  \centering
  \includegraphics[width=\textwidth]{figures/results_plan_flags_legend.pdf}\\[-.5em]
  \subfloat[Time to Solution]{\includegraphics[width=0.5\textwidth]{figures/results_plan_flags_a.pdf}\label{fig:plan_flags_a}}
  \hfill
  \subfloat[Time for Forward Transform only]{\includegraphics[width=0.5\textwidth]{figures/results_plan_flags_b.pdf}\label{fig:plan_flags_b}}
  \caption{\fftw{} on Intel E5-2680v3 CPU computing \texttt{powerof2} 3D single-precision real-to-complex in-place forward transforms.}
  \label{fig:fftw_plan_flags}
\end{figure}
\pause
\vspace{-1em}
\begin{itemize}
\item mean and standard deviations over 10 runs per signal shape
\item \mc{FFTW_MEASURE} imposes considerable time for planning
\item \mc{FFTW_MEASURE} beats \mc{FFTW_WISDOM_ONLY} for large input signals
  % this is actually a contradiction to that wisdoms use patient mode to get more optimal plans -> weird fftw behavior probably due to multi-core use
\end{itemize}
\end{frame}


\begin{frame}{Results}{fftw vs. clFFT vs. cuFFT}
\begin{figure}[!tbp]
  \centering
  \includegraphics[width=\textwidth]{figures/results_non_power_of_2_legend.pdf}\\[-1em]
  \subfloat[Time to Solution]{\includegraphics[width=0.5\textwidth]{figures/results_non_power_of_2_b_total.pdf}\label{fig:non_power_of_2_b}}
  \hfill
  \subfloat[Time for Forward Transform only]{\includegraphics[width=0.5\textwidth]{figures/results_non_power_of_2_a.pdf}\label{fig:non_power_of_2_a}}
  % changed order of a and b to be consistent with previous slide
  \caption{\fftw{} and \clfft{} on Intel E5-2680v3 CPU with 24 threads versus \cufft{} on P100 GPU computing single-precision real-to-complex out-of-place forward transforms of 3D shapes.}
  \label{fig:non_power_of_2}
\end{figure}
\end{frame}

%\setbeamertemplate{background canvas}{\includegraphics[width=\paperwidth]{rshiny.png}}
\begin{frame}{Results on the web}
  \begin{center}
    \includegraphics[width=.8\textwidth]{rshiny.png}\\[20pt]
    \href{https://github.com/mpicbg-scicomp/gearshifft/tree/master/rshiny}{mpicbg-scicomp/gearshifft/tree/master/rshiny}
  \end{center}

\end{frame}

%\resetbeamertemplate

\begin{frame}{Summary}{}
  
  \begin{itemize}
  \item \gearshifft{} enables in-depth analysis of FFT libraries
  \item there is a $<3\%$ latency overhead due to the timer objects
  \item when benchmarking fix the CPU/GPU clock if possible
  \item found bugs in each library
    % clfft: CPU, complex-to-real fails for 4096-FFT (not fixed yet)
    % clfft: memory leak
    % fftw: multi-core slowdown at planning stage, 2d/3d out-of-place not part of default wisdoms [undocumented]
    % cufft: recreating plans after context has been destroyed crashed in cuda 7.5 (fixed in cuda8)
  \end{itemize}
\end{frame}

\begin{frame}{Discussion}{}
  fftw has long planning times, but finds better plans:
  \begin{itemize}
  \item useful when plan is computed once and reused thousands of times
  \item not intended for multi-core and NUMA systems
  \item has some pitfalls in compiling fftw, plan-rigors and wisdom generations
    % (sse2 vs. avx, openmp vs. mpi vs. threads, do not mix float and double , fixes used number of cores to wisdom, patient vs measure and their vulnerability to multi-core slow-down issue)
  %\item multi-core increases planning time a lot, still same FFT performance as single-core is chosen by fftw
  \end{itemize}

  clFFT works on GPU and CPU.
  \begin{itemize}
  \item fast planning, but low FFT performance
  \item cannot handle arbitrary radices
    %(only combos of powers of 2,3,5,7,11,13)
  \end{itemize}

  cuFFT works on Nvidia GPUs.
  \begin{itemize}
  \item fast planning, average FFT performance on small FFTs
  \item outperforms on large FFTs
  \end{itemize}
  
\end{frame}


\end{document}
