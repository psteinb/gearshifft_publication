\documentclass[t,10pt,hyperref={
  %pdfpagemode=FullScreen,
  pdftitle = {gearshifft},
  pdfsubject = {gearshifft},
  %linktocpage=true,
  pdfborder={0 0 0},
  colorlinks=true,
  urlcolor=red,
  citecolor=red,
  linkcolor=red,
  pdfauthor={Peter Steinbach, Matthias Werner}
  }
]{beamer}
\usetheme{custom}
\def\resetbeamertemplate{\setbeamertemplate{background canvas}{ }}
\let\Tiny=\tiny
%\usepackage{lmodern}
\usepackage{tikz}
\usetikzlibrary{matrix}
\usetikzlibrary{calc}
\usetikzlibrary{positioning}
\usetikzlibrary{arrows}
\usetikzlibrary{trees}
\usetikzlibrary{backgrounds}
\usetikzlibrary{mindmap}

\usepackage[utf8]{inputenc}
\usepackage[T1]{fontenc}
\usepackage{csquotes}
\usepackage{ifthen}
\usepackage[capitalize]{cleveref} % after hyperref

\usepackage{amsmath,amstext,amsthm,array,booktabs}
\usepackage{caption}
\usepackage{xcolor}
\usepackage{graphicx}
\usepackage{subfig}
\graphicspath{{../}}
\usepackage{colortbl}
\usepackage{listings}
\usepackage{dirtree}

\usepackage[%per=slash,
%            decimalsymbol=comma,
            locale=US,
            ]{siunitx}

%%%%%%%%%%
\providecommand\thispdfpagelabel[1]{}


\lstset{literate={~} {$\sim\,$}{1}}
\lstset{upquote=true}
\lstset{language=C++,
  basicstyle=\small\fontfamily{fvm}\selectfont,%\ttfamily,
% 	columns=fullflexible,
  columns=fixed,
  keepspaces=true,
%       keywordstyle=\color{black}\bf,
%       stringstyle=\color{red}\ttfamily,
  showstringspaces=false,
  commentstyle=\color[rgb]{0.3,0.3,0.3}\textit,
%       morecomment=[l][\color{gray}]{\#},
  identifierstyle=\color{black},
  morekeywords={size_t},
  emph={__global__,__device__},
        emphstyle=\textit,
        numbers=left,
  numberstyle=\small\color{gray},
  breaklines=true,
  numbersep=5pt,
  xleftmargin=.19in,
}
% %%%%%%%%%%%%%%%%%%%%%%%%%%%%%%%%%%%%%%%%%%%%%%%%%%%%%%%%%%%%%%%%%%%%%%%%%%%%%%%

% \mode<presentation>{%
% \setbeameroption{show notes}
% }

\newcolumntype{M}{>{$\displaystyle}l<{$\vspace{1ex}}}
\title[gearshifft]{\texorpdfstring{%
    gearshifft\\[.5em]\Large The FFT Benchmark Suite\\ for Heterogeneous Platforms}}

\author[Steinbach, Werner]{\texorpdfstring{%
    \begin{minipage}[t]{.49\textwidth}%
      \centering%
      Peter Steinbach\\[.5em]
      {\footnotesize{Max Planck Institute of Molecular Cell Biology and Genetics\\
          Dresden, Germany}}\\
      {\small{\url{steinbac@mpi-cbg.de}}}
    \end{minipage}%
    \begin{minipage}[t]{.49\textwidth}%
      \centering%
      Matthias Werner\\[.5em]
      {\footnotesize{Center for Information Services and High Performance Computing\\
          TU Dresden, Germany}}\\
      {\footnotesize{\url{Matthias.Werner1@tu-dresden.de}}}
    \end{minipage}}{The Author}%
}

\date{June 20, 2017}

%%%%%%%%%%%%%%%%%%%%%%%%
\tikzset{invisible/.style={opacity=0},
      visible on/.style={alt=#1{}{invisible}},
      alt/.code args={<#1>#2#3}{%
        \alt<#1>{\pgfkeysalso{#2}}{\pgfkeysalso{#3}} % \pgfkeysalso doesn't change the path
      }
    }
\tikzset{class/.style={inner sep=5pt,font=\footnotesize}}
\newcommand{\pclass}[5][]{
\ifthenelse { \equal {#1} {} }
 {\node[class] (#5) at (#3,#4) {#2};}
 {\node[class] (#5) at (#3,#4) {%
\begin{tabular}{c}\scriptsize{<<#1>>}\\{\textbf{#2}}\end{tabular}%
};}
}
\newcommand{\pclassfill}[6][]{
\ifthenelse { \equal {#1} {} }
 {\node[class,rounded corners,fill=#6] (#5) at (#3,#4) {#2};}
 {\node[class,rounded corners,fill=#6] (#5) at (#3,#4) {%
\begin{tabular}{c}\scriptsize{<<#1>>}\\{\textbf{#2}}\end{tabular}%
};}
}

% https://tex.stackexchange.com/questions/263345/listings-and-verbatim-like-environments-in-a-tikz-node-using-newcommand
\newsavebox\mybox

\newcommand{\tkzgearshifft}{
%align=center,rounded corners,inner sep=5pt,rectangle,draw,
\begin{tikzpicture}
\tikzset{gr1/.style={}}
\tikzset{bts/.style={draw,circle,inner sep=3pt,fill=white}}
\tikzset{btc/.style={draw,circle,inner sep=3pt,fill=black}}

% fft parameters
\visible<4>{
\begin{scope}[yshift=3.75cm,xshift=-5.5cm,every node/.style={anchor=west,align=left,font=\small,draw,rectangle,rounded corners,fill=white}]

\node[fill=green!50] at (0,0) {cuFFT};
\node at (0.2,-0.4) {float, \ldots};
\node at (0.4,-0.8) {1024x1024, \ldots};
\node[fill=black,text=white] (mplc) at (0.6,-1.2) {Inplace\_Real, \ldots};

\end{scope}
}

% tree
\visible<5->{
\begin{scope}[yshift=3.5cm,xshift=-3.6cm]
\node[bts,fill=green!50] (b0) at (0,0) {};
\node[bts] (b10) at (-0.5,-0.6) {}; \draw (b10) -- (b0);
\node[bts] (b11) at (0.5,-0.6) {}; \draw (b11) -- (b0);
\node[btc] (b20) at (-0.75,-1.3) {}; \draw (b20) -- (b10);
\node[btc] (b21) at (-0.25,-1.3) {}; \draw (b21) -- (b10);
\node[btc] (b22) at ( 0.25,-1.3) {}; \draw (b22) -- (b11);
\node[btc] (b23) at ( 0.75,-1.3) {}; \draw (b23) -- (b11);
\node[font=\small] at (0, 0.4) {{\textbf{Boost}} Test Suites};
\node[font=\small] at (0,-1.7) {{\textbf{Boost}} Test Cases};

\end{scope}}

% 
\begin{scope}[xshift=1.75cm]
\begin{scope}

\visible<3->{
\pclass{{\textbf{Benchmark}}}{-2}{4}{b}
\pclass[Functor]{BenchmarkSuite}{-2}{3.2}{bs}
\pclass[Functor]{BenchmarkExecutor}{-2}{2.1}{be}
}
\pclassfill[Functor]{FFT}{-2}{1.0}{fft}{yellow}

\end{scope}
\begin{scope}

\pclassfill[Realisation]{Context}{1.5}{2.5}{ctx}{green!50}
\pclassfill[Realisation]{FFTClient}{1.5}{1.1}{impl}{green!50}
\visible<2->{
\pclass[Singleton]{Application}{1.5}{4}{app}
% \draw[rounded corners, dashed] (ctx.north west) rectangle (impl.south east);
% \draw[rounded corners] (ctx.north west) rectangle (ctx.south east);
% \draw[rounded corners] (impl.north west) rectangle (impl.south east);
\draw[rounded corners] (app.north west) rectangle (impl.south east);
}
\end{scope}
\end{scope}

\visible<-5>{
\matrix[
 minimum height=1.5em,
 matrix of nodes,
 row sep=-\pgflinewidth,
 ampersand replacement=\&,
 column sep=-\pgflinewidth,
 text depth=2.5ex,
 text height=1.5ex,
 text width=3.0em,
 align=center,
 nodes in empty cells,
 row 1/.style={nodes={fill=black!15,draw=black!40,text=black,thick,rectangle,draw,minimum width=2.5em,font=\scriptsize}}
]
(mf) at (0,-0.5) {
allocate \&
init\linebreak forward \&
|[gr1]| upload \&
|[gr1]| execute\linebreak forward \&
init\linebreak inverse \&
|[gr1]| execute\linebreak inverse \&
|[gr1]| download \&
destroy\\
};
}

\begin{scope}[thick]
\visible<-5>{
\draw (mf-1-1.north west) ++(-0.75em,0.5em) coordinate (ctl) -- ([xshift=0.75em,yshift=0.5em]mf-1-8.north east) coordinate (cr);
\draw[dotted] (ctl) -- ++(-2ex,0); \draw[dotted] (cr) -- ++(2ex,0);
\draw (mf-1-1.south west) ++(-0.75em,-0.5em) coordinate (cl) -- ([xshift=0.75em,yshift=-0.5em]mf-1-8.south east) coordinate (cr);
\draw[dotted] (cl) -- ++(-2ex,0); \draw[dotted] (cr) -- ++(2ex,0);
% total time
\draw[dashed,|-|] (mf-1-1.south west) ++(0,-1.5em) -- ([yshift=-1.5em]mf-1-8.south east) node[pos=0.5,fill=white,font=\small\itshape,align=center,anchor=north] {total time\\(measured separately)};
% (mf-1-1.south west) -- ++(0,-1.5em) -| (mf-1-8.south east) node[pos=0.25,fill=white,font=\small] {total time};
}

\visible<3->{
% \draw[-latex] (b) -- (bs);
% \draw[-latex] (fft) -- (ctl-|fft);
\draw[black!80] (b.south west) -- (b.south east);
\draw[black!80] (bs.south west) -- (bs.south east);
\draw[black!80] (be.south west) -- (be.south east);
\draw[-angle 60] (b) -- (app.west|-b);
}
\visible<4>{
  \draw[black!80, dashed] (b.south west) -- ++(-1em, 1.25em) -- ++(-5em,0); % test suite marker
  \draw[black!80, dashed] (bs.south west) -- ++(-1em,-1.5em) -- ++(-4em,0); % test suite marker
}

\visible<2->{
  \draw[-angle 60] (app) -- (ctx);
}
\draw[-angle 60] (impl) -- (ctx.south);
% \draw[densely dashed,-angle 90] (be.east) -| (impl.north);
% \draw[densely dashed,-open triangle 60] (impl) -- (fft.east|-impl) node[midway] (q) {};
\draw[-angle 90] (fft) -- (impl.west|-fft) node[midway] (q) {};
\visible<-5>{\draw[] (q.center) -- (ctl-|q);}
\end{scope}

\visible<6->{
  \node[align=left, anchor=north] (mpl) at (-1,-0.65) {Compile-time loops for precision and transform type, i.\,e.:\\[.5em]\usebox\mybox};
  \draw[-latex] (mpl) -- (mplc);
}
\end{tikzpicture}
}


%%%%%%%%%%%%%%%%%%%%%%%%%%%% 
\definecolor{arccl}{rgb}{0.45,0.45,0.45}
\definecolor{mc1}{rgb}{0.0, 0.0, 1.0}
\definecolor{mc2}{rgb}{0.0, 0.5, 0.5}
\definecolor{mc3}{rgb}{0.5, 0.0, 0.0}
\newcommand{\mgraymidrule}{\arrayrulecolor{arccl}\midrule\arrayrulecolor{black}}
%%%%%%%%%%%%%%%%%%%%%%%%%%%%%%%%%%%%%%%%%%%%%%%%%%%%%%%%%%%%%%%%%%%%%%%%



\newcommand{\gearshifft}{\texttt{gearshifft}}
\newcommand{\fftw}{\texttt{fftw}}
\newcommand{\cufft}{\texttt{cuFFT}}
\newcommand{\clfft}{\texttt{clFFT}}
\newcommand{\nvidia}{Nvidia}
\newcommand{\mc}[1]{\lstinline!#1!}
\newcommand{\iu}{{\mathrm{i}\mkern1mu}}

%% Todo: replace dirtree with gearshifft infos
%%        what is templated, what is runtime

\begin{document}
\frame[plain]{\titlepage}

\begin{frame}{Fields of Fast Fourier Transformations}
  \centering
  \begin{tikzpicture}
    \begin{scope}[every node/.style={align=center, anchor=north, font=\bfseries}, xscale=1.15, yscale=2.6]
      \node[fill=yellow,inner sep=5pt,draw,rectangle,rounded corners] at (0,-0.5) {\textbf{FFTs}};
      % https://commons.wikimedia.org/wiki/File:JPEG_compression_Example.jpg
      \node at (-3, 0.0) {\includegraphics[width=0.25\textwidth]{jpeg-compression.jpg}\\ compression};
      % https://imagej.net/Multiview-Reconstruction
      \node at ( 3, 0.0) {\includegraphics[width=0.25\textwidth]{imagej.jpg}\\ biology};
      % https://commons.wikimedia.org/wiki/File:Deep_learning.png
      \node at (-3, 1.0) {\includegraphics[width=0.21\textwidth]{deep-learning.png}\\ machine learning};
      % https://en.wikipedia.org/wiki/Trader_%28finance%29#/media/File:Philippine-stock-market-board.jpg
      \node at ( 3, 1.0) {\includegraphics[width=0.25\textwidth]{Philippine-stock-market-board.jpg}\\ financial math};
      % https://de.wikipedia.org/wiki/Sinc-Funktion#/media/File:Si_sinc.svg
      \node at ( 0, 1.0) {\includegraphics[width=0.25\textwidth]{sin-cos.png}\\ signal processing};
      % https://en.wikipedia.org/wiki/Interferometry#/media/File:USA.NM.VeryLargeArray.02.jpg
      \node at (-3,-1.0) {\includegraphics[width=0.25\textwidth]{usa-nm-vla.jpg}\\ astronomy};
      \node at ( 0,-1.5) {\Large\textbf{\ldots}};
      % https://en.wikipedia.org/wiki/Geology_applications_of_Fourier_transform_infrared_spectroscopy#/media/File:FTIR_spectrum.jpg
      \node at ( 3,-1.1) {\includegraphics[width=0.25\textwidth]{FTIR_spectrum.jpg}\\ geology};
    \end{scope}
  \end{tikzpicture}
\end{frame}


\begin{frame}{Introduction}{Fourier Transformations}

\vfill
  \begin{equation}
    \label{eq:dft}
    \text{\textbf{DFT:} }\quad X[k] = \sum_{j=0}^{n-1} x[j]\cdot\exp\left(\frac{-2\pi \iu jk}{n}\right),\quad x,X\in\mathbb{C}^n
  \end{equation}
\vfill  
  \begin{itemize}
  \item fast implementation of the discrete Fourier transform (DFT) \eqref{eq:dft}
  \item forward transform: time domain $\Rightarrow$ frequency domain
  \item factorization of $n$ and recursive decomposition smaller DFTs are computed with $\mathcal{O}(n\log n)$

    \begin{itemize}
    \item Cooley-Tukey: Radix-2 DFTs
    \item Stockham's formulations avoid incoherent memory access
    \item Bluestein's algorithm allows arbitrary and mixed radices
    \end{itemize}

  \end{itemize}
\vfill

\end{frame}

\begin{frame}{Introduction}{FFTs and their parameters}
  \centering  
  \begin{tikzpicture}[root concept/.style={rectangle},
    small mindmap]
%level 1 concept/.append style={sibling angle=20}, small mindmap]

    \path[concept color=white,text=black]
    node[concept] {\Large{Fast Fourier Transforms}}
    [grow=down]
    child[concept color=red, text=white] { node[concept] {\textbf{library}\\ cuFFT, clFFT, fftw, \ldots} }
    child[concept color=green] { node[concept] {\textbf{transform}\\ real, complex} }
    %child[concept color=brown] { node[concept] {\textbf{specifics}\\ rigors} }
    child[concept color=black,text=white] { node[concept] {\textbf{radices}\\ $2^n$, mixed} }
    child[concept color=cyan] { node[concept] {\textbf{dims}\\ $1$, $2$, $3$} }
    child[concept color=gray,text=white] { node[concept] {\textbf{hardware}\\ CPU, \\GPU} }
    child[concept color=orange] { node[concept] {\textbf{memory}\\ inplace, outofplace} }
    child[concept color=blue,text=white] { node[concept] {\textbf{precision}\\ single, double, \ldots} };

  \end{tikzpicture}

  \vfill
  \pause

  \setbeamertemplate{itemize items}[triangle]
  \begin{itemize}
  \item Which FFT implementation works best on what hardware?
  \item What hardware is best for my transform configuration?
  \item What has changed between library versions?
  \item \ldots
  \end{itemize}
  \setbeamertemplate{itemize items}[circle]

\end{frame}

\begin{frame}{gearshifft}

\vfill
      \begin{center}\Large
        \href{https://github.com/mpicbg-scicomp/gearshifft}{\textbf{github.com/mpicbg-scicomp/gearshifft}}
      \end{center}

  % \begin{columns}[c]
  %   \begin{column}{.8\textwidth}
  %   \end{column}
    
  %   \begin{column}{.2\textwidth}
  %     \\[-.5em]
  %   \end{column}

  % \end{columns}
\vfill
  \begin{itemize}
  \item free and open-source benchmark suite for FFT libraries
  \item licensed under Apache License 2.0
  \item community-ready
  \item vendor independent 
  \item standardized output format
  \item open and extensible architecture written in C++14
  \end{itemize}
\vfill
\pause
\begin{center}
  \includegraphics[width=.4\textwidth]{gearshifft_logo_img_100.png}
\end{center}

\vfill
\end{frame}

\begin{frame}[fragile]{gearshifft}{rational}

  Simplified Workflow for a single FFT benchmark of an FFT lib\\
  (e.g. 1024-FFT real-inplace single-precision):\\[.5em]
\begin{tikzpicture}[every node/.style={anchor=north west,align=left}]
    \node (a) {\textbf{FFT/lib operations}\\[.5em]
\begin{lstlisting}[numbers=none]
init_context();

init_plan();
exec_plan();
destroy_plan();

destroy_context();
\end{lstlisting}};

\node[visible on=<3->] (b) at (3.75,0) {
  \textbf{+ timer}\\[.5em]
      \begin{lstlisting}[numbers=none]
// ...

timer_start();
init_plan(...);
val = timer_stop();

// ...
  \end{lstlisting}};

\node[visible on=<4>] (c) at (7.5,0){
  \textbf{+ store times}\\[.5em]
     \begin{lstlisting}[numbers=none]
// ...

timer_start();
init_plan(...);
val = timer_stop();
store(val, PlanInit);

// ...
  \end{lstlisting}};
\end{tikzpicture}

\only<2>{
\begin{itemize}
\item data transfers not listed here
\item context has application lifetime
\item \gearshifft{} uses round-trip FFTs
\item workflow is templated by \mc{FFT.hpp}, implemented by cuFFT et. al
\end{itemize}}

\only<3>{
\begin{itemize}
\item time FFT operations
\item \gearshifft{} templates and reuses timer objects\newline
  (developer can provide own timer class)
\end{itemize}}

\only<4>{
\begin{itemize}
\item time values are collected
\item \gearshifft{} dumps results to csv in the end
\end{itemize}}

\end{frame}
  

% code used in tkzgearshifft
\begin{lrbox}{\mybox}
\begin{lstlisting}[numbers=none]
for( T_Precision : {float, double})
 for( extent : {32, 64x32, 16x16x16})
  for( T_FFT : {Inplace_Real, Outplace_Complex})
   // instantiate and run benchmark instance
\end{lstlisting}
\end{lrbox}


\begin{frame}[fragile]{gearshifft}{Basic Framework in C++/Boost}
  \centering
  \tkzgearshifft
\vspace{-2em}
\only<1>{  
  \begin{itemize}
  \item \mc{FFTClient} implements FFT workflow template class
  \end{itemize}
  }
\only<2>{  
  \begin{itemize}
  \item Application controls context object and results
  \end{itemize}
}
\only<3>{  
  \begin{itemize}
  \item Boost UTF driven generation of benchmark instances
  \end{itemize}
  }
\only<4>{  
  \begin{itemize}
  \item Boost test suites are generated by lib, precision and extents, the Boost test case is a specialized FFT client
  \end{itemize}
  }
\only<5>{  
  \begin{itemize}
  \item the benchmark tree is traversed by Boost \ldots
  \end{itemize}
  }
\end{frame}

\begin{frame}[fragile]{gearshifft}{Command-line}
  \begin{itemize}[<+->]
\item 
benchmark clFFT on CPU; load extents.csv; dump results to result.csv
\begin{lstlisting}[numbers=none,language=bash]
./gearshifft_clfft -f ../config/extents.csv
                   -o result.csv -d cpu
\end{lstlisting}

\item Command-line supports wildcards from Boost options
\item 
benchmark clFFT on GPU with extent=1048576, float and out-of-place real transforms
\begin{lstlisting}[numbers=none,language=bash]
./gearshifft_clfft -e 1048576
                   -r */float/*/Outplace_Real
# -r <Lib> / <Precision> / <Extents> / <Transform>
\end{lstlisting}
\end{itemize}  
\end{frame}


\begin{frame}{gearshifft}{Libraries}
  cuFFT 
  \begin{itemize}
  \item Nvidia's CUDA implementation of FFTs for Nvidia GPUs only
  \item supports arbitrary and mixed radices
  \end{itemize}

  clFFT
  \begin{itemize}
  \item AMD's OpenCL implementation of FFTs
  \item optimized for AMD GPUs, but also works on CPUs and other compute devices
  \item supports mixed radices 2, 3, 5, 7, 11 and 13
  \end{itemize}

  fftw
  \begin{itemize}
  \item for CPUs only
  \item code 20 years old, not designed for modern multi-core CPUs
  \item search-depth for optimal plan can be controlled by rigor flags
  \item precomputed plans can be stored to ``wisdoms''
  \end{itemize}
  
\end{frame}




\begin{frame}{Results}{Validation gearshifft cufft vs. standalone cufft}
\begin{figure}[!htb]
  \centering
  \includegraphics[width=\textwidth]{figures/results_validate_cufft_legend.pdf}\\[-.5em]
  \subfloat[1024-point FFT]{\parbox[b]{0.46\textwidth}{%
    \includegraphics[width=0.5\textwidth]{figures/results_validate_cufft_a.pdf}\\
    \includegraphics[width=0.5\textwidth]{figures/results_validate_cufft_c.pdf}
    }\label{fig:tts_verify_a}}
  \hfill
  \subfloat[16777216-point FFT]{\parbox[b]{0.5\textwidth}{%
      \includegraphics[width=0.5\textwidth]{figures/results_validate_cufft_b.pdf}\\
      \includegraphics[width=0.5\textwidth]{figures/results_validate_cufft_d.pdf}
    }\label{fig:tts_verify_b}}
  \caption{\textit{gearshifft} (\cufft{}) vs. \textit{standalone} \cufft{} --  multiple timer objects \& one timer object (\textit{standalone-tts}) -- single-precision in-place real-to-complex round-trip FFTs on the K80.}
  \label{fig:verify_cufft}
\end{figure}
\end{frame}


\begin{frame}{Results}{fftw plan-rigor flags}
\begin{figure}[!htbp]
  \centering
  \includegraphics[width=\textwidth]{figures/results_plan_flags_legend.pdf}\\[-.5em]
  \subfloat[Time to Solution]{\includegraphics[width=0.5\textwidth]{figures/results_plan_flags_a.pdf}\label{fig:plan_flags_a}}
  \hfill
  \subfloat[Time for Forward Transform only]{\includegraphics[width=0.5\textwidth]{figures/results_plan_flags_b.pdf}\label{fig:plan_flags_b}}
  \caption{\fftw{} on Intel E5-2680v3 CPU with \mc{FFTW_ESTIMATE}, \mc{FFTW_MEASURE} and \mc{FFTW_WISDOM_ONLY} computing \texttt{powerof2} 3D single-precision real-to-complex in-place forward transforms.}
  \label{fig:fftw_plan_flags}
\end{figure}
\end{frame}


\begin{frame}{Results}{fftw vs. clFFT vs. cuFFT}
\begin{figure}[!tbp]
  \centering
  \includegraphics[width=\textwidth]{figures/results_non_power_of_2_legend.pdf}\\[-1em]
  \subfloat[Time for FFT]{\includegraphics[width=0.5\textwidth]{figures/results_non_power_of_2_a.pdf}\label{fig:non_power_of_2_a}}
  \hfill
  \subfloat[Time to Solution]{\includegraphics[width=0.5\textwidth]{figures/results_non_power_of_2_b_total.pdf}\label{fig:non_power_of_2_b}}
  \caption{\fftw{} and \clfft{} on Intel E5-2680v3 CPU with 24 threads versus \cufft{} on P100 GPU computing single-precision real-to-complex out-of-place forward transforms of 3D shapes.}
  \label{fig:non_power_of_2}
\end{figure}
\end{frame}

\setbeamertemplate{background canvas}{\includegraphics[width=\paperwidth]{rshiny.png}}
\begin{frame}[plain]
\end{frame}

\resetbeamertemplate

\begin{frame}{Summary}{}
\end{frame}


\end{document}

%%% Local Variables:
%%% mode: latex
%%% TeX-master: t
%%% End:
