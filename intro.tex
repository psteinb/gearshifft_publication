Fast Fourier Transforms (FFTs) are at the heart of many signal processing and phase space exploration algorithms. Examples for their substantial usage come from image restoration in life sciences \cite{preibisch2014efficient,schmid2015real}, phase space reduction in weather forecasts \cite{maronga2015parallelized} to machine learning \cite{collobert2011torch7,jia2014caffe,abadi2016tensorflow} to just name a few.

As input data grows in size and variety both of experimental hardware \cite{huisken2004optical} and simulation outputs \cite{maronga2015parallelized}, input data on the order of Gigabytes to FFT libraries becomes the standard. With the advent of graphics processing units (GPUs) for scientific processing computing around the beginning of the 21st century and the subsequent availability of general purpose programming paradigms to program these \cite{du2012cuda}, vendor-specific and open-source libraries to perform FFTs on accelerators emerged (cuFFT \cite{nvidia2010cufft} by nVidia, open-source clFFT \cite{clfft}) to offer performance which supersedes traditional high-performance implementations running on standard Central Processing Units (CPUs) such as the open-source fftw library \cite{FFTW97,FFTW05} or the intel specific MKL \cite{intel2007intel}.

However, a review of the top ten sites listed of the fastest worldwide computer installations (top500 \cite{meuer2011top500}) shows that the used hardware is by far not homogenous. As this trend can be observed in practice even more, library architects and domain specialists are confronted with the question: which FFT implementation works best on what hardware? which hardware is to be suggested for optimal performance? which library offers the lowest overhead at best usability, long-term support and performance? As none of the aforementioned points can be answered in a straight forward and simple way, we propose an open-source benchmark package called \texttt{gearshifft} \cite{gearshifft_github} that allwos to benchmark available state-of-the-art FFT libraries in an reproducible, automated, open and vendor-independent fashion.



